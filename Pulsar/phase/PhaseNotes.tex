\documentclass[12pt]{article}
\usepackage[pdftex]{graphicx}
\usepackage[inner=3cm,outer=2cm]{geometry}
\usepackage{url}
\setlength{\topmargin}{0in}
\setlength{\headheight}{0in}
\setlength{\headsep}{0in}
\setlength{\textheight}{7.7in}
\setlength{\textwidth}{6.5in}
\setlength{\oddsidemargin}{0in}
\setlength{\evensidemargin}{0in}
\setlength{\parindent}{0.15in}
\setlength{\parskip}{0.10in}

\begin{document}
%\input ../../macros/banner.tex
\rightline{V1.0~~\today}
\bigskip
\bigskip
\bigskip
\begin{center}
{\Large \bf Notes on Phase Determintation for Pulsar Signals}
\end{center}

\section{General}
The phase of a pulsar timing signal as a function of time is given by
\begin{equation}
\phi(t) = \phi_0 + \int_{t_o}^{t} f(t') dt' ,
\end{equation}
where $\phi_0$ is the phase at time $t=t_0$ and $f(t)$ is the pulsar frequency as a function of time.   To zeroth order, $f(t) = f_0 = 1/T = {\rm constant}$, where $T$ is the period of the pulsar as seen in the barycenter of the solar system.   There are, however, small $\mathcal{O}(10^{-5})$ corrections stemming  from the Doppler shift associated with the Earth's orbital and rotational motion.   Thus
\begin{equation}
f(t) = f_0 \left(1 + \frac{v_D(t)}{c}\right),
\end{equation}
where $c$ is the speed of light and $v_D$ is the motion of the telescope projected onto the line of sight to the star under observation.   
 
\end{document}